Line 2 is the C include file for R-GMA. It must be included
to make any use of the provided R-GMA C library.

Line 4 is the name of the virtual database.

Lines 5--6 is the create table statement defining the column names and
types.

Line 7 is the authorization rule in the form
predicate:credentials:action. This can be a list of rules.

Line 9 defines a pointer to an RGMAException. Exceptions are handled uniformly
by the API. The last arument for all calls capable of failing is a pointer to a
pointer to an RGMAException.  It is essential that after each call a check is made. 

Line 10 contacts the schema and adds the table definition.

Line 11 checks to see if an RGMAException object has been created.

Line 12 report the error

Line 13 frees the exception object. This must be done each time an
RGMAException object is created by a failed call.

\subsection {Alter Table}

If the schema is modified resources using it will close down and send a temporary
exception so that the client is unaffected. Any resource will then be recreated
and if it needs a named tuplestore will find that the schema has changed and
will modify the stored table to match. 

If a table is dropped and recreated then the application code may find the table
missing and get a permanent exception .

Not all possible schema changes can be
carried out so an ALTER TABLE call is provided to support only those changes
that can be made reliably and to ensure that at all times the table exists and
so only a temporary exception will be generated as described above.
If a named tuplestore cannot make the change correctly to the stored data it
will simply destroy the old table and create a new one. This will never happen if the
ALTER TABLE operation on the schema is used. Changes to the columns of the
primary key or the addition of a "not null" column are not permitted. Types of
columns may not be changed except between CHAR and VARCHAR or to increase the
size of CHAR or VARCHAR. If a column is removed - it will be removed from the
database table. If a column is added - it will be added and set to null for all
existing tuples.

The ALTER TABLE and ALTER VIEW facilities are not provided as part of the API
but only from the command line tool.




\subsection {Drop Table}
\label{sec:dropTable}
In order to drop a table replace line 10 from the create table example
with:
\begin{verbatim}
RGMASchema_dropTable(vdb, "userTable", &exception);
\end{verbatim}
This will remove the table definition from the schema. Within a few
minutes any producer and consumer resources still using the table will
be destroyed.