Line 3 includes the header file for the R-GMA C API.

Lines 9--10 define the SQL query

Lines 11-16 try to create the consumer with a continuous query of \texttt{SELECT
userId, aString, aReal, anInt MeasurementTime FROM default.userTable}. In this case we have a
continuous query which will start with available published tuples from
the last 10 minutes (600 seconds) and receive any new tuples published. The
parameter with a value of 300 ensures that the query stops after 300 seconds.

Lines 17--38  retrieve all of the results for the query. While there is
data available it will loop without delay. If there is no data
currently available it will sleep for 2 seconds before trying
\texttt{pop} again. The 2000 is the maximum number of tuples to be
returned at once. Note the freeing of the tuple set to avoid memeory leaks.

Line 39 closes the consumer and frees memory. The exception is not checked as all
that could be done would be to print a message and the memory from any created
exception is not a concern as the program will exit.

\subsubsection {One-off Queries}
For one-off queries, either history or latest, it is preferable to
check to see if the query aborted. This can be achieved by looking to
see if \texttt{RGMAConsumer\_hasAborted()} is true after either kind of
consumer loop.

This can be the result of hitting the timeout (300 seconds in this
case) or making an explicit \texttt{RGMAConsumer\_abort()} call. Note that
continuous queries \emph{only} stop by one of these means so there is
no point in checking in that case.

\subsubsection{Resilient Consumer Example}
This example illustrates how to write a resilient consumer according
to the recommendations in section \ref{sec:advice}.
The pattern of the code is very similar to the earlier resilient primary
producer example.

\input{ResilientConsumer-c}

Lines 10--11 define the SQL query.

Lines 12--27 loop until the consumer creation is succesful or creates a
permanent exception.

Lines 28--56 loop popping data and printing it out. If tuples are present they
are printed out otherwise there is a two second pause. Warnings are
printed whether or not tuples are present. Exceptions are handled in the usual
way.
