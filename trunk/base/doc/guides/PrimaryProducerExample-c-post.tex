Line 2 is the C header file for R-GMA. It must be included
to make any use of the provided R-GMA C library.

Line 5 declares a pointer to an \texttt{RGMAPrimaryProducer}.

Lines 12--17 create the producer. The storage is of type memory. Memory
storage has no name so this is followed by NULL. The producer is configured to 
only answer continuous queries. Note the error handling which always follows
the same pattern. In this example as the program quits on all errors no
attention is paid to freeing resources.

Lines 19--24 declare the intention of publishing to the specified
table. The first argument identifies the primary producer --- this is true for
all calls except the \texttt{RGMAPrimaryProducer\_create}. The next
pair of arguments specifies the table name and the predicate. The
table must already be known in the schema and the predicate may be used
to specify that you are publishing a specific subset of the complete
table. The next two arguments are the retention periods (see
\ref{sec:rp}: the history one stating how long tuples will be kept for
\emph{continuous} and
\emph{history} queries and the latest one for \emph{latest} queries.
The error checking looks just the same as for the
createPrimaryProducer. 

Lines 25--31 construct and insert a tuple. The error handling is again
the same as before. The latest retention period is specified as 0. This means
that the value specified when the table was declared will be used.

Lines 32--36 close the producer and free the memory associated with the
producer. Each call to an RGMAXXXX\_create should be paired with an
RGMAXXXX\_close or RGMAXXXXdestroy to avoid memory leaks.


\subsubsection{Resilient Primary Producer Example}

This example illustrates how to write a resilient producer according to the
recommendations in section \ref{sec:advice}. The Primary Producer publishes
information periodically every 30 seconds. If there is any temporary exception
the operation is repeated 60 seconds later. Permanent expceptions are fatal. In
this example  when dealing with the temporary exceptions care is taken to free
the exception to avoid a memory leak.

\input{ResilientPrimaryProducer-c}

Line 17 loop until the operation (creation of a primary producer) is succesfull
or there is a permanent exception. 

Lines 18--19 create the primary producer.

Lines 20--22 break out of the loop if succesful.

Lines 23--25 give up on a permanent exception.

Lines 27--30 for a temporary exception, free the exception and sleep for 60
seconds

Lines 32-46 are just like lines 17-31 but this time a table is declared.

Lines 47-64 are similar to lines 17-31 however instead of breaking out of the
loop for a succesful operation the code continues to loop publishing more data.
