\section{Mediator Subsystem}
\label{sec:mediator}

The mediator acts as an intermediary between the Consumer/Secondary Producer and the Registry,
as illustrated in the picture below.


The Mediator is not  exposed as a service.

\subsection{Concepts}
\subsubsection*{Plan}

A Plan describes the producers a consumer needs to contact to answer its query. It
consists of a list of PlanEntry structures, together with an optional warning string.
An empty string is assumed to mean ``no warning''.

\subsubsection*{PlanEntry}

A plan entry is a query to send to a Producer. It consists of the \texttt{ProducerDetails} object,
the SQL SELECT statement, a start time and an end time for the query.

\subsubsection*{PlanInstruction}

A plan instruction describes a modification to a plan. There are two types:

\begin{itemize}
\item An \texttt{ADD} instruction adds a PlanEntry to a Plan.
\item A \texttt{REMOVE} instruction removes a PlanEntry from a Plan.
\end{itemize}

Both types of instruction may have a warning which will be attached to the
modified query plan if it does not already have one.

\subsection{Operations}
\subsubsection{getPlansForQuery}

This is the main mediation operation. It is called by a consumer when its query is started
and is responsible for registering the consumer (for a \texttt{CONTINUOUS} query) and 
returning the query plan (the producers that must be contacted to answer the query).

\textbf{Continuous queries}

A single plan consisting of all relevant primary producers that can answer continuous
queries. If the query has an associated time interval 
and not all the relevant producers have a sufficient HRP to respect it, a 
warning will be attached to the plan. Continuous queries must be \textit{simple}, as
a result only one table need be considered (although VDB union is possible, these have
been removed at an earlier stage).

\texttt{getMatchingProducersForTables} is called on the registry instance for
the VDB in the query. This has the side-effect of registering the consumer in the
specified VDB. The matching producers are looped through and if the producer
supports continuous queries and is a primary producer, added to the list. If the
producer does not have a sufficiently long HRP, a flag is set to generate a
warning. Each producer is sent the query. 

\textbf{Latest/History queries}

Loop through all VDBs referenced in the query and get all matching producers for
the query from the registry instance for this VDB then update the list of producers, adding producers which are not in the list and
          adding vdb-table details to those that are already in the list.
  
  From the list of producers, extract a list of complete producers, i.e. those
        secondary producers which publish all the tables in the query, which support the
        query type and have no predicate (or a predicate which is complete with respect to the
        query).
  If at least one complete producer was found, form a separate plan for each one. Attach a
        warning to any plan if the query is a history query and the secondary producer has an
        insufficient HRP. Return this list of plans.
        
  If no complete producers were found, but primary producers do exist, create a fallback 
        plan consisting of all primary producers which publish all the tables and support the 
	query type. Attach a warning if there are other primary producers which are not included
        in this list, or if there is more than one primary producer and the query is complex
        (since complex queries cannot be answered completely by merging results from separate
        producers).
         
  If no primary producers exist, return an empty plan with no warning - there is no
        information in the system to answer the query so an empty plan is correct.

\textbf{Static queries}

Loop through all VDBs referenced in the query and get all matching producers for
the query from the registry instance for this VDB then update the list of producers, adding producers which are not in the list and
adding vdb-table details to those that are already in the list.
  
Create a plan consisting of all on-demand producers which publish all the tables. 
Attach a warning if there are primary producers which are not included
in this list, or if there is more than one on-demand producer and the query is complex


\subsubsection{refreshPlan}

This operation is called by continuous consumers to update their plan and
reaffirm their registration. It checks for any new relevant producers at the same
time. In principle the consumer should get notification directly from new
producers, but this method provides a fallback solution if this message is lost.

Get all matching producers for the query from the registry instance for the VDB.
This has the side-effect of re-registering the consumer.

If the plan consists only of secondary producers, continue to the next VDB,
ignoring the list of returned producers.

If the plan contains primary producers, check that each returned continuous
primary producer is included in the plan (either directly or via a secondary
producer). For any that are not included, create an 'add' instruction for the
producer. Attach a warning to the instruction if the producer's HRP is
insufficient to answer the query.

\subsubsection{addProducerToPlan}

This is called when a continuous consumer receives notification from a new,
potentially relevant producer.

If the producer is relevant and a  primary producer and if the producer is not
already covered by the plan (by a secondary producer) then  return an 'add'
instruction for the producer. Attach a warning to the instruction if the
producer's HRP is insufficient to answer the query.
 
\subsubsection{removeProducerFromPlan}

This is called when a consumer receives notification that a potentially relevant
producer no longer exists, or when any kind of consumer determines that a
producer in its current plan is not functioning. Both these forms of notification
are assumed to be more reliable than whether or not the producer exists in the
registry.

If the producer is currently part of the plan and is a primary producer return a
'remove' instruction for the producer.

If the producer is currently part of the plan and is a secondary producer we are
likely to need a new plan so:

Call \texttt{getPlansForQuery} to obtain a list of new plans. Remove any plans
that include the removed producer (it may still be in the registry but we trust
an explicit removal more).

For each plan, form a set of add/remove instructions that will change the current
plan into the new one. Choose the smallest set of instructions that results in a
new plan with no warning. If all the new plans have warnings, choose the smallest
set of instructions.
 
\subsubsection{closePlan}

 If the query is continuous call \texttt{unregisterContinuousConsumer} on
  the registry for the VDB referenced in the query.


