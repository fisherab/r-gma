Lines 0--8 are the various \texttt{include} statements.

Lines 9--15 are a set of \texttt{using} statements to avoid clutter
in the code.

Line 18 set up initial state for the secondary producer.

Lines 21--23 define the parameters and create the secondary
producer. In this case we have specified a logical database
(cppExample) providing support for latest queries.

Line 24--26 declare a table that the secondary producer will deal
with. The predicate (line 24) is an empty string meaning that this
secondary producer will collect and republish the whole table. The
history retention period is set to 2 hours. This means that tuples
will be available until they are 2 hours old - these tuples will be
made available to continuous queries. In addition tuples will be
available for latest queries. The latest retention period is a
property of the individual tuple as defined at the primary producer.

Line 27 obtain the termination interval from the R-GMA server.

Lines 28--32 keep the secondary producer alive. We must issue
an \texttt{showSignOfLife} call more often than the termination
interval. In this case the sleep is for one third of the termination
interval. It is possible that the \texttt{showSignOfLife} call may
fail if it is unable to contact the service, or the service is unable
to locate the remote resource. In this case the loop will be exited.

Lines 33--41 report any exceptions that may occur.

Lines 42--51 shuts down the secondary producer.

Lines 52--53 deletes the secondary producer.

N.B. A user can only have one producer using a specific storage
location at a time.

Notice that if the secondary producer does fail it could be one third
of the termination interval before the program detects this and exits,
so you need to set the retention period and sleep parameter according
to your needs.

\subsubsection{Avoiding a permanent connection to a Secondary Producer}
If you don't want to keep the process running continuously you could
leave the secondary producer running after disconnecting from the API
and then periodically run a job which reconnects.  This is what the 
\texttt{rgma-sp-manager} tool does as explained in
Section~\ref{sec:rgma-sp-manager}.

