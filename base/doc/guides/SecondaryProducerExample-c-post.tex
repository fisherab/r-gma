Line 3 includes the usual R-GMA header file for C.

Lines 8--13 obtain the server defined termination interval and divide this by 3
to compute a suitable interval between making showSignOfLife calls. In the code
below any exception will cause the program to terminate. If the code chose to
try again after a temporary execption then this computation would allow one
message to fail to be sent without losing the secondary producer.

Lines 14-19 create the secondary producer with named database storage and
support for continuous and latest queries.

Lines 20--25 declare a table that the secondary producer will deal with. The
predicate is an empty string meaning that this secondary producer will collect
and republish the whole table. The history retention period is set to 2 hours.
This means that tuples will be available to continuous queries until they are 2
hours old (the secondary producer has been created without support for history
queries). In addition tuples will be available for latest queries. In this case
however the latest retention period is a property of the individual tuple as
defined at the primary producer. Note the call to SecondaryProducer\_close
before exiting following an exception. This ensures that a new
secondary producer can be started immediately with the same named database
storage.
Lines 26--35 keep the secondary producer alive.
This program will continue to run for ever unless there is a problem.
Notice that if the secondary producer does fail it will not be noted until
showSignOfLife is called again, so you need to set the retention period 
and sleep parameter according to your needs.

\subsubsection{Avoiding a permanent connection to a Secondary Producer}
If you don't want to keep the process running continuously you could
leave the secondary producer running after disconnecting from the API
and then periodically run a job which reconnects.  This is what the 
\texttt{rgma-sp-manager} tool does as explained in
Section~\ref{sec:rgma-sp-manager}.

